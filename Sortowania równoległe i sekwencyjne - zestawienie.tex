\documentclass[11pt]{article}
\usepackage[utf8x]{inputenc}
\usepackage{polski}
\usepackage{amsthm}
\usepackage{listings}
\usepackage{amsmath}
%Gummi|065|=)
\title{\textbf{Sortowania równoległe i sekwencyjne}}
\author{Dorota Kapturkiewicz\\
		Wiktor Kuropatwa\\
		Karol Różycki}
\date{}
\begin{document}

\maketitle

\section{Cel projektu}
Celem pniższego projektu jest porównanie efektywności i czasu działania kilku algorytmów sortowania z różnym stopniem zrównoleglenia.

\section{Zaimplementowane algorytmy}
Projekt zaostał zaimplementowany w języku C++ (w standardzie c++ 11).
Do fragmentów wielowątkowych użyliśmy technologii OpenMP. \newline
Dana praca porównuje działania następujących algorytmów:
\begin{itemize}

\item merge sort 
\begin{itemize}
\item w pełni sekwencyjny 
\item równoległe wywołania rekurencyjne sortowania, sekwencyjne złączanie
\item równoległe sortowanie i złączanie
\end{itemize}

\item bitonic sort
\begin{itemize}
\item w pełni sekwencyjny 
\item zrównoleglone wywołania rekurencyjne i porównywanie wartości w bitonic merge'u
\end{itemize}

\item radix sort
\begin{itemize}
\item w pełni sekwencyjny 
\item zrównoleglony prefix sum, przepisywanie wartości i przygotowywanie danych (xor)
\end{itemize}

\end{itemize}


\section{Przeprowadzone testy}


\section{Wyniki}

\section{Wnioski}

\section{Źródła}


\end{document}
